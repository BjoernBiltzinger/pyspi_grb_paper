% Define document class
\documentclass[twocolumn]{aa}

% Filler text
\usepackage{caption}
\usepackage{subcaption}
\usepackage{float}
%\usepackage{natbib}
% Begin!
\begin{document}

% Title
\title{Fitting Synchrotron Models to Gamma-Ray Burst Data from INTEGRAL/SPI}

% Author list
\author{
  Björn Biltzinger
  \and
  Jochen Greiner
  \and
  Michael Burgess
  \and
  Siegert
}

\institute{Max-Planck-Institut fur extraterrestrische Physik, Giessenbachstrasse 1, D-85748 Garching, Germany \label{mpe}}
\date{Received 18 December, 2019; accepted 22 May, 2020}

\label{firstpage}
% Abstract with filler text
\abstract {
  In the last years it has been shown, that it is needed to fit physical models to Gamma-Ray Burst (GRB) data to get more insights into their emission mechanism. For the energy range between ~ 10 keV to ~ 10 MeV Fermi/GBM has been the main instrument to analyze GRBs. In this paper we show how we can add INTEGRAL/SPI to this analysis. INTEGRAL/SPI covers a similar energy range like Fermi/GBM and has a very good energy resolution of 2.2 keV at 1.3 MeV. Therefore INTEGRAL/SPI is an ideal instrument to precisely measure the curvature of the spectrum. We present PySPI, a new analysis software we developed to analyze GRB data from SPI, and apply it to the GRB 120711A. We show that PySPI improves the analysis of SPI data, for GRBs, compared to the official analysis software and that the GBM and the SPI data of this GRB can be fitted well with a physical synchrotron model and that combining both, reduces the allowed parameter space. This demonstrates that SPI can play an important role in GRB model fitting.
}

\keywords{ (stars:) gamma ray bursts -- methods: data analysis --
  methods: statistical}

\maketitle

% Main body with filler text
\section{Introduction}


%\bibliographystyle{aasjournal}
%\bibliography{biblio}

\end{document}
